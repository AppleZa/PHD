\documentclass{ResearchProposal} % The class file specifying the document structure

\thesistitle[Optional Short Title]{PHD Thesis}
\author{Kyle \textsc{ Spindler}}
\supervisor{Dr. Julian \textsc{Kunkel}} % Your prospect supervisor's name (if known already), leave empty if you are looking for one
\university{University of Reading} % The university you apply for
\department{Department of Computer Science} % The department's name
\group{\href{http://hps.vi4io.org}{High-Performance Storage}} % The research area/group
\keywords{Microservice, Serverless, HPC} % Use a few describing the thesis better


\addbibresource{example.bib}

\startMain

\textit{Please also check \url{http://www.reading.ac.uk/computer-science/dcs-PhD-programmes.aspx}.}


\section{Motivation}
Software Architecture invovles considering multiple characteristics such as separation of concerns, quality attributes (maintainability, scalability, loose coupling, high cohesion etc...) and architectural styles. Some architecture styles are more suited for performance while others are better at maintainability and loose coupling like microservices. Microservices is a very popular architecture that is used in many domains because of the benefits it offers.

\medskip

Scientic codes suffer from good software engineering practices. HPC and store applications are typically tightly coupled to utilize the available resources efficiently. While it is claimed that this provides the best performance, the benefit and drawbacks of alternative software architectures for HPC software is not thoroughly investigated. Microservices, for example, provide a scalable architecture and ease the software development process by providing separation of concerns by applying techniques from Domain Driven Design. When deciding a software architecture not only performance and scalability matters, but also flexibility and maintainability. 

\medskip

In this regard, the HPC community struggles to recruit sufficient developesr to keep up with the development of software which can often be seen in important utility tools. For example, existing tools for pre/post-processing of HPC workflows and the analysis of HPC data are typically not the main focus of scientists and developers; hence, they are implemented in a way that shows limited scalability, i.e. are executed sequentially in bash scripts.


\section{Research question}
\textit{The main research question(s) that you want to address.}

Understand the impact of modern day software architectures (microservices, event driven) has on HPC and particularly the climate/weather domain 

The goal of this thesis is to see if HPC applications and storage systems can be redeveloped using modern day software architecture such as microservices with minimal or no overhead while gaining the benefits from the loosely coupled architecture.

\begin{enumerate}
	\item What parts of the HPC application could benefit from microservices or other software architectures?
	\item What 
\end{enumerate}

Example: The goal of this thesis is to understand and optimize the performance behavior for large-scale data accesses in the domain of climate and weather.

\textit{Now split the research goal into questions}

\smallskip

This covers the research questions:
\begin{enumerate}
	\item What workflows are limited by I/O?
	\item Which I/O operations are typically performed?
	\item Which optimizations are beneficial for the workflows on HPC systems?
\end{enumerate}


\section{Related work}
\textit{How your thinking builds on any previous work.}

\medskip

Relevant work can be classified into: a) LaTeX studies, b) performance analysis in HPC, ....

\paragraph{LaTeX studies.} It has been shown that blabla \citep{lamport1994latex}.

\paragraph{Performance analysis.}

\section{Research methodology}
\textit{What research methodology or techniques you may need to use}

\section{Required infrastructure}
\textit{What facilities you are likely to require to conduct your research.}


\medskip

This research requires a supercomputer with more than 100 nodes to run experiments on.

\section{Workplan}
\textit{How the research can be completed in the time available. Provide a rough sketch over the runtime of your PhD}

The following sketches the workplan for the different years of the PhD.

\paragraph{First year:} setup of work environment, researching related work, writing the chapters introduction and related work of the thesis.

\paragraph{Second year:}

\paragraph{Third year:}

\proposalAppendix

Add here any appendix, if needed

\printbibliography[heading=bibintoc]

\label{LastPage}

\end{document}
