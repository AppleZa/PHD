\documentclass{ResearchProposal} % The class file specifying the document structure

\thesistitle[Optional Short Title]{MyThesis}
\author{MyFirstname \textsc{MyLastname}}
\supervisor{Dr. Julian \textsc{Kunkel}} % Your prospect supervisor's name (if known already), leave empty if you are looking for one
\university{University of Reading} % The university you apply for
\department{Department of Computer Science} % The department's name
\group{\href{http://hps.vi4io.org}{High-Performance Storage}} % The research area/group
\keywords{X, Y, Z} % Use a few describing the thesis better


\addbibresource{example.bib}

\startMain

\textit{Please also check \url{http://www.reading.ac.uk/computer-science/dcs-PhD-programmes.aspx}.}


\section{Motivation}
\textit{What makes the research topic of interest and importance}


\section{Research question}
\textit{The main research question(s) that you want to address.}

\textit{Provide one sentence that gives an overview of the topic, how would you describe your thesis to a computer scientist?}

Example: The goal of this thesis is to understand and optimize the performance behavior for large-scale data accesses in the domain of climate and weather.

\textit{Now split the research goal into questions}

\smallskip

This covers the research questions:
\begin{enumerate}
	\item What workflows are limited by I/O?
	\item Which I/O operations are typically performed?
	\item Which optimizations are beneficial for the workflows on HPC systems?
\end{enumerate}


\section{Related work}
\textit{How your thinking builds on any previous work.}

\medskip

Relevant work can be classified into: a) LaTeX studies, b) performance analysis in HPC, ....

\paragraph{LaTeX studies.} It has been shown that blabla \citep{lamport1994latex}.

\paragraph{Performance analysis.}

\section{Research methodology}
\textit{What research methodology or techniques you may need to use}

\section{Required infrastructure}
\textit{What facilities you are likely to require to conduct your research.}


\medskip

This research requires a supercomputer with more than 100 nodes to run experiments on.

\section{Workplan}
\textit{How the research can be completed in the time available. Provide a rough sketch over the runtime of your PhD}

The following sketches the workplan for the different years of the PhD.

\paragraph{First year:} setup of work environment, researching related work, writing the chapters introduction and related work of the thesis.

\paragraph{Second year:}

\paragraph{Third year:}

\proposalAppendix

Add here any appendix, if needed

\printbibliography[heading=bibintoc]

\label{LastPage}

\end{document}
