\PassOptionsToPackage{hyphens}{url}
\documentclass[compress,aspectratio=169]{beamer}

\usetheme{Reading}

\usepackage[official]{eurosym}
\usepackage{multirow}
\usepackage{units}

% \usepackage{todonotes} % if you want to make annotations

% Where are graphics to be found
\graphicspath{{../assets/}{./fig/}}


\title{Professional Talks with LaTeX}
\subtitle{-- Work in progress -- }
\authorURL{https://hps.vi4io.org}
\author{\underline{Julian M. Kunkel}, Author 2, Author 3} % Always mark the first author

% if you want to use a logo in the footer:
\authorFooter{Julian M. Kunkel \hspace{0.3cm} \includegraphics[height=1em]{hps-footer}}
\institute{Computer Science Department}
\venue{The name of the venue}

\groupLogo{\includegraphics[width=2cm]{hps-logo}}
\date{2018-05-08}
\titleLogo{\includegraphics[width=16.5cm,height=3cm]{background}
% can span the whole width
\\[-1em]
% If you use images from other sources, always give the reference and license:
\tiny \HREFColor{darkgray}{http://hps.vi4io.org}{Image under CC-BY-4 license}
}
% Members of the university may use:
% http://www.reading.ac.uk/imagebank

% The number of slides, if you want to provide them
%\slideNumber{22}

\newcommand{\comment}[1]{}

\begin{document}

\begin{frame}[plain]
	\titlepage
\end{frame}

\begin{frame}{Outline}
	\tableofcontents[subsectionstyle=hide/hide]
\end{frame}


\section{Introduction}

\subsection{Motivation}
\begin{frame}{Motivation}
	\begin{block}{Goal: Professional presentations}
		\begin{itemize}
				\item Logical structure
				\item Good design
				\item Self-speaking slides
		\end{itemize}
	\end{block}

	\begin{block}{Purpose of this slide deck}
		\begin{itemize}
			\item Illustrates the use of LaTeX Beamer
			\item Provides an unofficial template for Reading staff
				%\begin{itemize}
					%\item Students: Please use the template ReadingStudent due to the copyright of the first page
				%\end{itemize}
			\item Describes principles how to design presentations
		\end{itemize}
	\end{block}
\end{frame}

\section{Structure}
\sectionIntro

\subsection{Subsections}
\begin{frame}{Subsections}
\begin{itemize}
	\item Subsections help to structure the talk
	\item Logical bundling of topics
	\item The template renders the color of the active subsection differently
\end{itemize}
\end{frame}


\section{Layout}
\sectionIntro

\begin{frame}{Layout}
\begin{itemize}
	\item xx
\end{itemize}
\end{frame}

\begin{frame}{Graphics and Figures}
\begin{figure}[tbp]
	\centering
	\subfloat[Subcaption A\label{fig:pm}]{\includegraphics[width=4cm]{hps-logo}}
	\qquad % create some space
	\subfloat[Subcaption B\label{fig:emp}]{\includegraphics[width=4cm]{hps-logo}}
	\caption{Some caption}
\end{figure}
\end{frame}

\section{Summary}

\begin{frame}{Summary}
	\begin{itemize}
		\item LaTeX is easy to use and produces appealing presentations
		\item Structure and layout
	\end{itemize}
\end{frame}

\backupbegin

\section{Backup Slides}
\begin{frame}{Backup Slides}
	\begin{itemize}
		\item It is good practice to prepare backup slides
		\item Benefit:
		\begin{itemize}
			\item If you talk too slow, you may show some extra information
			\item They may answer detailed questions
		\end{itemize}
	\end{itemize}
\end{frame}

\backupend
\end{document}
